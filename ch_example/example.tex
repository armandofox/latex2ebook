% optional first arg to Chapter overrides what will appear in running
% heads in PDF version.
\chapter[Example]{Example Chapter}

This shows how you can make an example chapter.

\section[Images]{Using Images}

Here's how you can include an image.  Images \textbf{must} be PDF
files.  If you choose not to use the \texttt{picfigure} macro, you will
have to add your own conditional macro to generate the correct HTML
markup.  The Makefile takes care of converting the PDF's to GIF images,
including layer flattening, alpha channel (transparency) removal,
scaling to a resolution and size appropriate for ebook readers, and
more.

\picfigure{ch_example/figs/example_figure.pdf}{fig:example_figure}{%
  This is the caption of the figure.  Note that you can reference the
  label \texttt{fig:example\_figure} on your text, and that the filename
  of the figure omits the \texttt{.pdf} extension.
}

\section{Sidebars}

Keep sidebars short.  They will be laid out in the margin for PDF but
due to ebook formatting limitations will appear inline for the ebook, so
textual placement of the sidebar should respect that.  The first
argument to the sidebar is the sidebar's title, which will appear in
bold.  
  \begin{sidebar}[-0.25in]{An example sidebar}
    \ldots{}looks like this.  Not too long or you'll have formatting
    pain in ebook-land.
  \end{sidebar}
Use the optional first argument to control positioning of the sidebar in
the PDF output; by default, the top edge of the sidebar box will line up
with the place the sidebar is referenced.  

More sophisticated is the ability to include a small PDF image as part
of a sidebar.  Again, in the ebook version it'll appear inline.

\section{Tables and Other Textual ``Figures''}

  
\section{Code}

The \texttt{lstlistings} package is used to format code.  It knows about
many languages' syntax, but inspect the \texttt{listingdefaults.tex}
file to tweak its behavior.  
\emph{Place each code example, however
  short, in its own separate file,} since postprocessing on code files
for ebook readers 
automatically adds line numbers (be sure your code files don't have
extraneous trailing blank lines).  

You can include code inline with the text like this:

\codefile{ch_example/code/hello_world.c}

You can also reference code as a figure, as
Figure~\ref{fig:code_example} shows.

\codefilefigure{ch_example/code/hello_world.c}{fig:code_example}{%
  Another way of referencing code, by letting \protect\LaTeX float the
  figure and give it a label as usual.
}



\section{Changebars}

\section{Integration With Pastebin}

\section{Screencasts}

\section{Web Links}

In the ebook version, web links are live; in the print version, an
endnote is used to associate a URI with the link text.  The endnote
appears in the To Learn More section at the end of the chapter.

You can include \weblink{http://saasbook.info}{plain old URI's} and also
URIs to Wikipedia articles whose name matches the URI, such as the
article on \w{client-server architecture}, as well as articles whose
name doesn't quite match the URI, such as the one on
\w[Peer to peer]{peer-to-peer architecture}.   Try typing the exact
phrase you want (including spaces, case-insensitive) into Wikipedia to
see which one to use.  Spaces in Wikipedia URIs are automatically
converted to underscores.

\begin{summary}
  \textbf{Summary:}

  A summary box can be used to recapture the highlights of the section
  or chapter.  \textbf{Very important:} if the summary box doesn't fit
  on a single screen on the ebook reader---and remember some ebook
  readers have small screens and the (human) reader can increase the
  font size---very bad things will happen.
\end{summary}

\section{Bibliography}

A \texttt{book.bib} file \emph{must} be present, so that you can cite
great stuff such as \emph{The Mythical
  Man-Month}~(\cite{brooks:manmonth}).  Beware: I did some tricks to
enable per-chapter bibliography, but not all bib styles work with those
tricks.  \texttt{natbib} seems to work so I used that.

\section{Exercises}



\begin{tolearnmore}

This section is generated by the \texttt{tolearnmore} environment.
You can change its name by changing the definition of the
\texttt{tolearnmore} environment in \texttt{macros.tex}.

Per-chapter bib entries are placed here after any text you provide (if
you want to override that, remove \texttt{putbib} macro from the
definition of the \texttt{tolearnmore} macro).  As well, any endnotes
such as URIs in the print version will appear after the end of the
environment. 

\end{tolearnmore}
